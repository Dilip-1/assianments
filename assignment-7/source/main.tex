\documentclass{article}
\usepackage[utf8]{inputenc}
\usepackage{enumitem}
\usepackage{karnaugh-map}

\title{EC2014,25}
\author{dilip}
\date{December 2020}

\begin{document}

\maketitle
\section{Question}
Minimised expression for $(x+y)(x+\overline{y})+\overline{(x\overline{y})+\overline{x}}$

\begin{enumerate}[label=(\alph*)]
    \item \textit{x}
    \item \textit{y}
    \item \textit{xy}
    \item \textit{x+y}
\end{enumerate}
\section{Answer}
let,
\begin{equation}
F=(x+y)(x+\overline{y})+\overline{(x\overline{y})+\overline{x}}
\end{equation}
\subsection{Truth Table}
\begin{table}[h]
\centering
\begin{tabular}{|c|c|c|c|}
\hline
\textit{\textbf{x}} & \textit{\textbf{y}} & \textbf{F} & \textbf{Minterm} \\ \hline
0                   & 0                   & 0          & -                \\ \hline
0                   & 1                   & 0          & -                \\ \hline
1                   & 0                   & 1          & $x\overline{y}$  \\ \hline
1                   & 1                   & 1          & $xy$             \\ \hline
\end{tabular}
\caption{Truth table for F}
\label{tab:my-table}
\end{table}
\section{K-Map}
\begin{karnaugh-map}[2][2][1][][]
    \maxterms{0,2}
    \minterms{1,3}
    \implicant{1}{3}
    \draw[color=black, ultra thin] (0,2) --
    node [pos=0.7, above right, anchor=south west] {$x$} 
    node [pos=0.7, below left, anchor=north east] {$y$}
    ++(135:1);
\end{karnaugh-map}
From the k-map we can conclude that the simplified expression for F is \begin{equation}
 F=x   
\end{equation}
\end{document}




